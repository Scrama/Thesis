I'm a header-weader!
\looks{like i have a tags}
and some strings
%
% --==[ F:\Projects\thesis\Тех файлы\27_Chernih_Kononov.tex ]==--
%
\title{Exact polynomial-time algorithm for the two-machine routing flow shop problem with a restricted transportation network
\thanks{This research was supported by RFBR grant 17-07-00513 and 18-01-00474.} }
%
\titlerunning{Short Title}  % abbreviated title (for running head)
%                                     also used for the TOC unless
%                                     \toctitle is used
%
\author{Ilya Chernykh\inst{1,2,3} \and
Alexander Kononov\inst{1,2} \and
Sergey Sevastyanov\inst{1,2}} 
%
\authorrunning{Ilya Chernykh, Alexander Kononov and Sergey Sevastyanov } % abbreviated author list (for running head)
%
%
\institute{Sobolev Institute of Mathematics, Novosibirsk, Russia\\
\email{idchern@math.nsc.ru, alvenko@math.nsc.ru, seva@math.nsc.ru}
 \and
Novosibirsk State University, Novosibirsk, Russia\\
Novosibirsk State Technical University, Novosibirsk, Russia \\
 }

\maketitle              % typeset the title of the contribution

\keywords{shop scheduling, routing, dynamic programming}
\\

We consider the routing flow shop problem being a generalization of the flow shop and the metric travelling salesman problems.
The jobs are located at nodes of some transportation network, and the machines travel on the network to execute the jobs in the flow shop environment. 
The machines are initially located at the same node (depot) and must return to the depot after completing all the jobs. 
It is required to find a non- preemptive schedule with the minimal makespan.

The routing flow shop problem was introduced by Averbakh and Berman \cite{AverbakhBerman}. 
It is strongly NP-hard even in the single machine case as it contains the metric TSP as a special case.
Yu et al. \cite{Yu} proved that the two-machine routing flow shop is ordinary NP-hard when the transportation network is a tree
with unbounded number of nodes, i.e. the number of nodes is a linear function on the number of jobs.

In our paper we consider the two-machine routing flow shop under the assumption that a transportation network is an arbitrary undirected graph with a constant number of vertices. 
We present an exact polynomial-time algorithm for this case based on dynamic programming. 
This stands in contrast to the complexity result for the two-machine routing open shop problem
that is NP-hard even on the two-node network \cite{ABC}.






\begin{thebibliography}{1}
\providecommand{\url}[1]{\texttt{#1}}
\providecommand{\urlprefix}{URL }

\bibitem{AverbakhBerman}
Averbakh, I., Berman, O.:  Routing two-machine flow shop problems on networks with special structure. Transportation Science 30, 303--314 (1996).
\bibitem{ABC} 
Averbakh I, Berman O, Chernykh I: The routing open-shop problem on a network: complexity and approximation. European Journal of Operational Research 173(2), 521--39 (2006).
\bibitem{Yu}
Yu W., Liu Z., Wang L., Fan T.: Routing open shop and flow shop scheduling problems. European Journal of Operational Research 213, 24-36 (2011).


\end{thebibliography}
%
% --==[ F:\Projects\thesis\Тех файлы\119_Romanova.tex ]==--
%
\title{Минимизация затрат  в задаче календарного планирования со
складируемыми ресурсами }
%
\titlerunning{Краткое название}  % abbreviated title (for running head)
%                                     also used for the TOC unless
%                                     \toctitle is used
%
\author{Анна А. Романова \inst{1,}\inst{2}, Татьяна А. Сергиенко \inst{1,}\inst{2}}
%
\authorrunning{Романова А.А., Сергиенко Т. А.} % abbreviated author list (for running head)
%
%
\institute{Омская юридическая академия, Омск, Россия,\\
 \and
Омский государственный университет им. Ф.М. Достоевского, Омск, Россия,\\
\email{anna.a.r@bk.ru} }

\maketitle              % typeset the title of the contribution

\textbf{Ключевые слова:} вычислительная сложность, календарное
планирование, динамическое программирование \\

В работе рассматривается следующая задача календарного
планирования. Имеется проект, состоящий из $n$ взаимосвязанных
работ, для выполнения которых требуются складируемые ресурсы $m$
видов. Взаимосвязь между работами задается ориентированным графом
$G=(V,E)$, где $V$ -- множество вершин-работ; дуга $(i,j)\in E$,
если работа $j$ не может начаться до завершения работы $i$. Для
каждой работы $j\in V$ известны ее длительность $p_{j} \in Z^+$ и
потребность $q_{jr}$ в ресурсе $r$ в каждый период времени
выполнения, $r=1,\ldots,m$. Для выполнения работ ресурс $r$  в
любой период времени можно приобрести по обычной цене
$C^{norm}_r$, если объем покупаемого ресурса не превышает
$V^{norm}_r$. При превышении этого уровня устанавливается новая
цена $C^{over}_r$ за единицу ресурса. Требуется определить
допустимое расписание выполнения работ проекта и план закупок
ресурсов, при которых суммарные затраты на приобретение ресурсов
минимальны. В работе рассматриваются случаи неограниченного и
ограниченного склада. В первом случае в каждый период времени
можно складировать любое количество ресурса. При ограниченном
складе излишек ресурса $r$, то есть ресурс, который предполагается
потратить позже, в каждый период времени не должен превышать
объема $V^{cont}_r$ склада. 
Ранее в ряде случаев установлена
сложность задачи. В частности, показано, что при ограниченном
складе как при $C^{over}_r>C^{norm}_r$, так и при
$C^{over}_r<C^{norm}_r$ задача NP-трудна, при неограниченном --
полиномиально разрешима \cite{romanova_R,romanova_PR}.
В настоящей работе продолжено исследование данной задачи.
Предложены модели целочисленного программирования для случая
ограниченного склада; разработаны алгоритмы динамического
программирования; выделены малотрудоемкие частные случаи.
\begin{thebibliography}{1}
\providecommand{\url}[1]{\texttt{#1}}
\providecommand{\urlprefix}{URL }

\bibitem{romanova_R}
Романова, А.А.: Исследование одной задачи календарного
планирования со складируемыми ресурсами. Динамика систем,
механизмов и машин : матер. VIII Междунар. науч.-техн. конф, с.
88--91.  Омск : Изд-во ОмГТУ (2012)
\bibitem{romanova_PR}
Пирогов, А.Ю., Романова, А.А.: О сложности одной задачи построения
расписания с минимальными затратами на приобретение ресурсов.
Россия молодая: передовые технологии -- в промышленность, \No 3,
с. 69-73 (2015)
\end{thebibliography}
\end{document}